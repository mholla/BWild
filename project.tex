% Original from Maria Holland
% Updated June 2020

\documentclass[11pt]{article}
\usepackage[english]{babel}
\usepackage{enumitem}
\usepackage[top=0.75in,bottom=0.75in,left=1in,right=1in]{geometry}
\usepackage{graphicx}
\usepackage[colorlinks,allcolors=blue]{hyperref}
\usepackage{multicol}
\usepackage{multirow}
\usepackage{xcolor}

\setlist{itemsep=1pt}
\setlength\parindent{0pt}
\setlength{\parskip}{6 pt}

\begin{document}

\begin{center}
AME 50572: Introduction to Biomechanics \, Fall 2019

{\Large Semester Project: Biomechanics in the Wild} 
\end{center}

\hrulefill \\

\section*{Overview}

In your semester project for this course, you will choose a biomechanics topic of interest to you, explore the scientific literature to learn about this topic, and then share what you learn to at least two audiences (biomechanics experts, your classmates, the general public, etc.) in at least two media (online blog, oral presentation, formal written report, etc.).   
% \begin{itemize}
% \item to the general public, in a post on \href{http://sites.nd.edu/biomechanics-in-the-wild/}{the course blog};
% \item to your classmates, in an oral presentation; and
% \item to the instructor, in a formal written report.
% \end{itemize}

\paragraph{Goals:}
The goals of this project are to explore a topic in biomechanics beyond those covered in class, 
to provide a starting point for exploration of the biomechanics literature in your area of interest, 
and to share your findings with both your peers in the class and the public at large.  

These goals connect to the course goals of ``identifying challenging questions across the breadth of biomechanics'', ``read, summarize, and evaluate scientific literature in biomechanics'', and ``communicating about biomechanics to a variety of audiences'', as well as to the 3rd and 7th ABET student outcomes, ``an ability to communicate effectively with a range of audiences'' and ``an ability to acquire and apply new knowledge as needed, using appropriate learning strategies''.

% This project spans several levels of cognitive skills, from comprehension (understanding original research papers), analysis (comparing and contrasting different studies), and evaluation (identifying limitations of the approaches). 

Additionally, in the long term, I hope that this project will prime you to notice biomechanics (and other subjects that you've only encountered in the classroom) `in the wild' --- that is, in the world around you --- and to talk about them with other people.

\paragraph{Project Parts:}
This semester-long project consists of several parts: 
\begin{enumerate}
\item topic ideation (30 points)
\item blog post (75 points)
\item lay feedback (5 points)
\item class presentation (75 points)
\item peer review (15 points)
\item final report (200 points)
\item for select volunteers: podcast or video (150 points)
\end{enumerate}

\paragraph{Topics:}
Your project may stem from a question that occurs to you that has something to do with biomechanics: 
Why is some hair curly?  
How long until a figure skater does a quintuple jump?  
Or you could choose a peer-reviewed journal article you find interesting.
It does not have to be recent. 
If you have questions about finding a source, it may be helpful to have a conversation with Dr. Holland during office hours.

In addition to the topic, you must identify at least three peer-reviewed articles on the topic.  At most one may be a review article.  
Solid journals that publish biomechanics papers include, but are not limited to: 
\begin{itemize} 
    \item \href{https://www.journals.elsevier.com/journal-of-biomechanics}{\it J Biomechanics} 
    \item \href{https://link.springer.com/journal/10439}{\it Annals of Biomedical Engineering} 
    \item \href{https://www.journals.elsevier.com/biomaterials}{\it Biomaterials} 
    \item \href{https://www.journals.elsevier.com/acta-biomaterialia}{\it Acta Biomaterialia} 
    \item \href{https://www.journals.elsevier.com/journal-of-the-mechanical-behavior-of-biomedical-materials}{\it J Mechanical Behavior of Biomedical Materials} 
    \item \href{https://link.springer.com/journal/10237}{\it J Biomechanics and Modeling in Mechanobiology}
\end{itemize}

\section{Ideation Assignment}
Complete a pre-project written assignment to help you brainstorm and guide your selection of a topic.  
\begin{enumerate}
\item Read at least four posts from \href{http://sites.nd.edu/biomechanics-in-the-wild/}{the class blog}, and write a 2-3 sentence summary for each one.

\item Picking one of the posts that you read, evaluate it according to the rubric at the end of this document.  Make at least one suggestion for improvement.

\item Select your project topic and write a 2-3 sentence description.  

\item List 3-5 papers that you may use for the project.  Include the author list (or First Author et al.), title, journal name, and a permalink (DOI, PMID, PMCID).  Indicate if they are original research or review papers.
\end{enumerate}

\section{Blog Post}
Write a few paragraphs summarizing your topic for the class \href{https://sites.nd.edu/biomechanics-in-the-wild/}{Biomechanics in the Wild blog}.  Keep in mind the following guidelines: 

\paragraph{Audience:} 
You are writing for the general public.  You have some flexibility in what this means: a reasonably-educated audience without any engineering background?  High school students?  Elementary school kids?%
\footnote{Various programs can assess the readability of text and offer suggestions on how to improve or simplify it - consider \href{http://www.hemingwayapp.com/}{The Hemingway Editor}, among others.}.
While your writing should be somewhat technical, also try to write in an engaging or journalistic style to encourage interaction and comments from readers.  

\paragraph{Content:} 
In the text, summarize the relevant biology and mechanics, clearly explaining any content that might be unfamiliar to the audience.  
Make sure to explain why this topic is important and why your readers should care about it!

\paragraph{Text:} 
Your post should be less than about 500 words.  
The words should be your own or, in rare cases, clearly denoted by quotation marks.  
Proofread carefully to fix grammar, spelling, and punctuation errors. 

\paragraph{Media:}
Indicate the source of all media.  
Additionally, consider the copyright issues that may arise as a result of the media you are using
{\bf (see the end of this document for more information on copyright)}.  For any media that you include, write a few sentences about why you believe it's okay to use (it may be helpful to use something like \href{https://www.lib.umn.edu/copyright/fairthoughts}{this fair use tool}), and submit them to the corresponding assignment on Sakai.
In order to better reach audiences that may have accessibility challenges, include \href{https://accessibility.umn.edu/core-skills/alt-text#Alt%20text%20in%20Canvas}{descriptive alt-text} for all media.  

\paragraph{Blogging:} 
Present your article in an engaging way that will attract internet audiences.  
Come up with an informative and engaging title. 
Include one (or more, as needed) images, photos, video, or figures to explain and clarify your topic, and pick an appropriate image to be featured in the post.
Select 1-5 informative tags (at most one new tag!) that would help guide interested readers to your post. 
Include links to sources (including at least one of the papers you read) and at least one source for additional reading on a related topic. 

\paragraph{Submission: }
Create your post in Wordpress, where you have been added as a contributor.  
\emph{Do not click publish; instead, when you are ready to submit your post, send it for review.}
When you are done, submit either a text or pdf copy of the post to Sakai, where it will be evaluated by Turnitin.
Plain text can be generated by pasting into a plain text editor like Notepad, Textwrangler, etc., and pasted directly into the text box in the Sakai assignment.  Alternately, print to PDF using your browser and attach that to the submission.
If you would like your post to be anonymous, email Dr. Holland and she will change the author name associated with the post.

\paragraph{Evaluation:} 
Your blog post will be evaluated based on the rubric found at the end of this document.  
After the initial submission of the blog post, points can be made up at 2/3 of their value by addressing the suggestions for improvement.


\section{Lay Feedback}
Reach out to a member of the target audience that you had in mind when you wrote your blog post.  Ask them to look at a draft of your post.  Have them write a few sentences answering the following questions: 
\begin{enumerate}
\item What about this article made you interested to know more about the topic?
\item What would make this article more interesting to you?
\item Where there any words or ideas that you found confusing? 
\item Do you have any other suggestions on how to improve the article?
\end{enumerate}


\section{Class Presentation}
Prepare a 5 minute presentation to the class summarizing your topic.  {\bf This time limit will be strictly enforced!}
Presentations will be scheduled shortly before Spring Break..   
The content should be similar to your blog post, but the audience for this component is obviously your classmates, who have more specialized technical knowledge.  
Images and \emph{short} videos are strongly encouraged, and citations are of course required (including at least one of the papers you read).   
Submit your slides to Google Drive \href{https://drive.google.com/drive/folders/1acV3yN3BzQR8oLNvmK3L5cnOHd7DqflE?usp=sharing}{here} in Powerpoint, Keynote, Google Slides, or PDF form by 8am on the day of your scheduled presentation.  Presentations will only be rescheduled due to excused absences.

\paragraph{Evaluation:} 
Your presentations will be evaluated based on the rubric found at the end of this document, including by your peers.

\section{Peer Review}
A preliminary draft of the final report will be due 1-2 weeks before the final due date.  
At that time, we will exchange drafts for peer review.
You will be responsible for reviewing two of your classmates' papers, and in turn you will receive feedback on your paper from two classmates.  

\paragraph{Giving Feedback:}
For each paper you read, fill out rubric at the end of this document, including comments for improvement.  Additionally, write a few sentences answering the following questions: 
\begin{enumerate}
\item What is the main question that paper is trying to answer?
\item What terms or concepts in this report were new to you? If you can, explain them.
\item Could the writer of this report have omitted anything to make it more concise? If yes, what?
\end{enumerate}

\paragraph{Receiving Feedback:}
Write a few sentences or a paragraph discussing the feedback you got and how you plan to use it to improve the final version.



\section{Final Report}
Write a technical report that discusses your topic, drawing on the papers (at least three) that you read.   
Your report should be no longer than 6 pages, including figures, with at least 1-inch margins and size 11 font.  See the \href{https://www.overleaf.com/read/rkqrcvbjnhng}{LaTeX template} on Overleaf; you can download all the files under Menu $\rightarrow$ Download Source.

This report should be similar to a review paper, although it is understood that the scope will be limited to only a few papers and not an exhaustive literature search.  
Thus, it should include a section identifying the need and/or potential for future research on the subject.
It should include methods as needed to understand the study and its results.  

\paragraph{Guiding Questions:}
As you prepare your report, the following questions (modified from The Cain Project\footnote{The Cain Project in Engineering and Professional Communication, Communication Resources. OpenStax CNX. Jul 24, 2008 http://cnx.org/contents/e651fc79-fae7-4912-babe-22464ffa379e@2.2.})
may help guide your analysis:
\begin{itemize}
\item What problem was investigated? What questions did the authors try to answer?
\item What was the primary motivation for investigating the problem?  Why were the investigators interested in the problem?  Why should the general public care?
\item What background information (context) is relevant to understanding the problem?
\item What are the paper's key points?
\item What method(s) or experimental design did the authors use?
\item Which of the results reported in the paper are the most significant?
\item Did the authors note any unexpected results? If so, were those results explained sufficiently?
\item What were the authors' conclusions?
\item What, if any, were the weaknesses of the paper (either self-identified by the authors or noticed by you)?  How can future studies overcome these weakenesses?
\item What terms or definitions are new to your audience? How can you explain them clearly?
\end{itemize}


\section{Podcast or Youtube Video}
A few selected volunteers may choose to present your topic in one of two alternative formats: a podcast episode or a Youtube vide.
This option would take the place of both the blog post and class presentation.  
{\bf If you are interested in this option, please talk to Dr. Holland (in person or via email) as early as possible.}  
If you are selected for this option, you will discuss on-campus resources for filming/recording.

In the ideation assignment, instead of reading posts from the class blog, listen to three science-based podcast%
\footnote{Some science podcasts include 
\href{https://soundcloud.com/biomechanics-on-our-minds}{BOOM: Biomechanics On Our Minds}, 
\href{http://thispodcastwillkillyou.com/}{This Podcast Will Kill You}, 
\href{https://www.alieward.com/ologies}{Ologies}, 
\href{https://www.wnycstudios.org/podcasts/radiolab}{Radiolab}, 
\href{https://www.sciencemag.org/podcasts}{Science Podcasts}, 
\href{https://www.nature.com/nature/articles?type=nature-podcast}{Nature Podcast}, 
\href{https://www.sciencefriday.com/listen/}{Science Friday}, 
\href{https://www.brainson.org/}{Brains On}
} 
episodes or watch three science-based Youtube videos%
\footnote{Some science channels include
\href{https://www.youtube.com/user/phdcomics}{PHD Comics}, 
\href{https://www.youtube.com/user/minutephysics}{minutephysics}, 
\href{https://www.youtube.com/user/grossscienceshow}{Gross Science}, 
\href{https://www.youtube.com/user/Kurzgesagt/videos}{In a Nutshell}, 
\href{https://www.youtube.com/user/scishow}{SciShow}, 
\href{https://www.youtube.com/user/AsapSCIENCE}{AsapSCIENCE}, 
\href{https://www.youtube.com/user/1veritasium}{Veritasium}, 
\href{https://www.youtube.com/user/talknerdytomeHP}{Talk Nerdy To Me}, 
\href{https://www.youtube.com/channel/UC7_gcs09iThXybpVgjHZ_7g}{PBS Space Time}, 
\href{https://www.youtube.com/user/SciAmerican/}{Scientific American}
}
.  Write a few sentences about elements that make them engaging and informative, and aspects that could be improved.  


\subsection{Podcast}
Your podcast episode should be at least 20 minutes long.  
It can optionally include a guest or interview.  
The inclusion of any music or sounds should follow the copyright considerations at the end of this document.
Your submission should include the podcast episode in MP3 format, shownotes, a transcript (in order to better reach audiences that may have accessibility challenges).


\subsection{Youtube Video}
Your video should be at least 3 minutes long.
The inclusion of any music, video, or images should follow the copyright considerations at the end of this document.
Your submission should include the podcast episode in AVI, MOV, or MP4 format, notes, and a transcript/captions (in order to better reach audiences that may have accessibility challenges).

\newpage
\section*{Guidelines on Academic Integrity for Projects} 

Integrity, academic and otherwise, is an important component of your education at Notre Dame.   
According to the Honor Code, academic integrity requires that you submit work that is your own.  

I am not reading your projects solely to understand the ideas in the original papers; if that were so, I would read those papers.  Instead, I am reading them to gauge your understanding of the paper and your ability to write about that understanding.

For this reason, it is very important that you use your own words in your writing.  This may be challenging, because the first step of writing about something in your own words is to understand it very well.  I suggest that you should not write while you have the source material in front of you.  Instead, take abbreviated notes (perhaps summarizing each paragraph in a single sentence) and write from your notes.  In general, if you can only write with the original source in front of you, you most likely do not have the level of understanding that you will need to write about it.  

MIT\footnote{\href{https://integrity.mit.edu/}{Academic Integrity at MIT}} and the Writing Center at the University of Wisconsin Madison\footnote{\href{https://writing.wisc.edu/Handbook/QuotingSources.html}{Quoting and Paraphrasing}} have some very helpful resources on academic integrity and avoiding plagiarism that I encourage you to consult.  I would emphasize that 
\begin{itemize}
\item you do not need to restate the entire paper. Be selective and highlight only the most important, most interesting, or most necessary information.  
\item changing words while leaving the content and structure otherwise the same is not an acceptable way to paraphrase (for examples, see UW's \href{https://writing.wisc.edu/Handbook/QPA/QPA_paraphrase.html}{ ``Successful vs. unsuccessful paraphrases''} or MIT's \href{https://integrity.mit.edu/handbook/academic-writing/avoiding-plagiarism-paraphrasing}{``Paraphrasing''})
\item excerpting whole sentences or paragraphs and including them in quotation marks with a citation is technically not plagiarism but it is generally not appropriate in scientific writing like this.
% https://integrity.mit.edu/handbook/academic-writing/avoiding-plagiarism-choosing-whether-quote-or-paraphrase
\item if you use figures from other sources (which you are encouraged to do!), you must properly attribute this.  For example, include in the caption ``(taken from [1])'', or ``(modified from [2])'' if you did something to the original image.  (Please note that additional restrictions apply for the Biomechanics in the Wild project, which will be publicly availble.)
\item figure and table captions should also be written in your own words
\end{itemize}

Summarizing, paraphrasing, and writing in your own words are important skills in a variety of occupations and activities.  Please take advantage of these projects as opportunities to strengthen these skills, and avail yourself of the resources you have available (including Turnitin, described below, and me).

\paragraph{TurnItIn:}
The projects in this course will be submitted to TurnItIn via Sakai.  TurnItIn compares submissions to a huge library of published material and other student submissions, and highlights phrases that are found elsewhere.

To help you develop your technical writing skills and give you formative feedback on your writing, you will be submitting drafts of your projects to Turnitin before submitting the final project.  Take the time to view the originality report of your drafts, and if TurnItIn identifies major similarities with the sources you used, you should revise your project before submitting the final version.  {\bf Note: Draft assignments have been made for the purpose of getting feedback and revising.  Do not submit to the actual assignment on Sakai until you are satisfied with the final version!!}  

To view the TurnItIn submission report:
\begin{enumerate}
\item Submit your file as usual into the proper assignment on Sakai.
\item You will receive an email notification from TurnItIn stating that your file was successfully submitted.
\item Wait about 10 minutes for the report to generate.
\item On Sakai, open the assignment and click the green/yellow/red box that is next to `TurnItIn Report'.
(Green indicates \textless 25\% similarity, yellow \textless 50\%, and red \textgreater 50\%.) 
\item A new tab will open from TurnItIn showing your similarity report.
\end{enumerate}
The column on the right shows, among other things, your similarity score (a single number, expressing the percentage of your writing found elsewhere) and a list of sources that were detected.  

You do not need to be concerned about every similarity found by TurnItIn; for example, if you reproduce a table of data it will probably appear as a match with the original source.  Matches in individual words and short phrases will often be unavoidable, but if significant parts of sentences or paragraphs are highlighted, you should revise.  

Total similarities should generally be below 10\%.  If the total similarity is over 25\%, or if any one source is responsible for 10\% or more of the similarity, you should revise your report substantially before submitting it to be graded.  If you have questions about how to do that, please reach out to me.  If final submissions exceed those levels, the project grade will reflect the amount of work it is possible to attribute to you as opposed to other sources.  Additionally, I may consider other consequences, up to and including an Honor Code Violation Report.  

\newpage
\section*{Copyright:}
Are you planning to use text, images, or other content that you did not create? 
If so, and the content was created after 1923, it is a good idea to assume that someone else holds the copyright.

Copyright is actually a bundle of rights that authors and/or creators have as copyright owners. 
If you are the copyright owner, you have the right to make copies of your work, distribute your work, make derivative works of your work, and perform and publicly display your work.

If you are not the copyright owner, one of your options is to use content that is in the public domain, which means that either a) the copyright has expired for that content or b) the author/creator has chosen to relinquish some or all of their rights to it. Good sources for such images include 
\href{https://search.creativecommons.org/search}{Creative Commons Search},
\href{https://commons.wikimedia.org/wiki/Main_Page}{Wikimedia Commons}, 
\href{https://www.flickr.com/creativecommons}{Flickr Creative Commons}, 
\href{http://www.everystockphoto.com/}{everystockphoto}, 
\href{https://pixabay.com/en/}{Pixabay}, 
\href{https://unsplash.com/}{Unsplash},
\href{https://morguefile.com/photos}{Morguefile},
\href{https://burst.shopify.com/}{Burst},
\href{https://www.rawpixel.com/}{Raw Pixel}, 
\href{https://www.stockvault.net/}{StockVault}, and
\href{https://smart.servier.com/}{Smart Servier Medical Art}.  

Another option for using copyrighted material is to use one of the limitations placed on the rights of copyright owners. 
This limitation, called \href{https://www.lib.umn.edu/copyright/fairuse}{fair use},  allows you to use copyrighted material without getting permission from the copyright owner under \href{https://www.lib.umn.edu/copyright/fairuse}{certain conditions}. 
The catch is that you, as the individual using the copyrighted material, must make the determination if your use of the material would be considered ``fair'' by applying the four factors (described \href{https://www.lib.umn.edu/copyright/fairuse}{here} and \href{https://www.law.cornell.edu/uscode/text/17/107}{here}). 
This does entail some risk, especially since this blog post will be publicly available and not confined strictly to the classroom or campus.

You should not pay for content to use in this assignment; if you do so, you may be outside the scope of copyright law and will be subject to any stipulations or restrictions mandated by the seller of the content.

For more information about the university's policies on copyright, please refer to the \href{https://www.nd.edu/copyright/}{University of Notre Dame’s Copyright page}. 

{\it This information is courtesy of Hesburgh Libraries (\href{mailto:hl-copyright-list@nd.edu}{hl-copyright-list@nd.edu}).}



\newpage
\newcommand{\category}[2]{\hline \parbox[t]{2mm}{\multirow{#1}{*}{\rotatebox[origin=c]{90}{{\bfseries{#2}}}}} }
\newcommand{\itemline}{& \\ \cline{2-4}}

\section*{Blog Post Rubric}
\renewcommand\arraystretch{2.0} 

\begin{tabular}{|c@{\hspace*{0.2cm}}|l|c|c|}
\hline 
& {\bf Criteria} & {\bf Value} & {\bf Points} \\
\category{5}{Content}
& Clear motivation for the topic & 10 \itemline
& Relevant biology and mechanics are presented & 10 \itemline
& Writing shows deep understanding of topic & 10 \itemline
& Topics unfamiliar to the audience are clearly explained & 10 \itemline
\category{5}{Text \& Media}
& Writing sparks interest and curiosity & 5 \itemline
% & Appropriate content for selected audience & 5 \itemline
% & Writing is original & 5 \itemline
& Media have been appropriately sourced and cited & 5 \itemline
& Descriptive alt-text for all media & 5 \itemline
& Informative and engaging title & 4 \itemline
& Correct grammar, spelling, and punctuation & 4 \itemline
\category{6}{Blogging}
& $\sim 500$ words & 2 \itemline
& 1+ image, photo, or video & 2 \itemline
& Appropriate featured image & 2 \itemline
& 1-5 appropriate tags & 2 \itemline
& Sources are mentioned and linked & 2 \itemline
& Link for additional reading & 2 \itemline
\hline
\end{tabular}

Suggestions for improvement: 


\newpage
\section*{Presentation Rubric}

\renewcommand\arraystretch{2.0} 

\begin{tabular}{|c@{\hspace*{0.2cm}}|l|c|c|}
\hline 
& {\bf Criteria} & {\bf Value} & {\bf Points} \\
\category{4}{Content}
& Clear motivation for the topic & 10 \itemline
& Relevant biology is presented & 10 \itemline
& Relevant mechanics is presented & 10 \itemline
& New topics are clearly explained & 10 \itemline
\category{8}{Presentation}
& $<5$ minutes & 5 \itemline
& Slides are clear and effective & 5 \itemline
& Sources are referenced appropriately for content and media & 5 \itemline
& Correct grammar, spelling, and punctuation & 5 \itemline
& Questions handled respectfully and knowledgeably & 5 \itemline
& Students in the back can follow the presentation & 5 \itemline 
& Peer assessment & 5 \itemline
\hline
\end{tabular}

Suggestions for improvement: 


\newpage
\section*{Podcast Rubric}

\renewcommand\arraystretch{2.0} 

\begin{tabular}{|c@{\hspace*{0.2cm}}|l|c|c|}
\hline 
& {\bf Criteria} & {\bf Value} & {\bf Points} \\
\category{6}{Content}
& Clear motivation for the topic & 15 \itemline
& Relevant biology and mechanics are presented & 20 \itemline
& Topics unfamiliar to the audience are clearly explained & 20 \itemline
& Content shows deep understanding of topic & 15 \itemline
& Content sparks interest and curiosity & 10 \itemline
& Content is organized logically and stays on topic & 10 \itemline
\category{9}{Podcast}
& Informative and engaging title & 5 \itemline
& $>20$ minutes & 5 \itemline
& Audio is clear and easy to understand & 10 \itemline
& 1+ extra content (guest interview, sound clip, etc.) & 10 \itemline
& Content from other sources has been appropriately sourced and attributed & 10 \itemline
& Mention of additional sources for more information & 5 \itemline
& Show notes enhance the audio content & 5 \itemline
& Transcript includes entire podcast & 5 \itemline
& All content has correct grammar, spelling, and punctuation & 5 \itemline
\end{tabular}

Suggestions for improvement: 


\newpage
\section*{Video Rubric}

\renewcommand\arraystretch{2.0} 

\begin{tabular}{|c@{\hspace*{0.2cm}}|l|c|c|}
\hline 
& {\bf Criteria} & {\bf Value} & {\bf Points} \\
\category{6}{Content}
& Clear motivation for the topic & 15 \itemline
& Relevant biology and mechanics are presented & 20 \itemline
& Topics unfamiliar to the audience are clearly explained & 20 \itemline
& Content shows deep understanding of topic & 15 \itemline
& Content sparks interest and curiosity & 10 \itemline
\category{9}{Video}
& Informative and engaging title & 5 \itemline
& $>3$ minutes & 5 \itemline
& Audio and video are clear and easy to understand & 10 \itemline
& Effective use of visual content & 10 \itemline
& Content from other sources has been appropriately sourced and attributed & 10 \itemline
% & Video contains appropriate balance of text and graphics & 5 \itemline
& Mention of additional sources for more information & 5 \itemline
& Description enhances video & 5 \itemline
& Video contains closed captions for all audio content & 5 \itemline
& All content has correct grammar, spelling, and punctuation & 5 \itemline
\end{tabular}

Suggestions for improvement: 



\newpage
\renewcommand\arraystretch{2.0} 

\section*{Final Report Rubric}

\begin{tabular}{|c@{\hspace*{0.2cm}}|l|c|c|p{3.0in}|}
\hline 
& {\bf Criteria} & {\bf Value} & {\bf Points} \\
\category{6}{Content}
& Clear motivation for the topic & 30 \itemline
& Methods presented as needed  & 30 \itemline
& Key results presented & 30 \itemline
& New topics are clearly explained & 30 \itemline
& Identification of limitations & 25 \itemline
& Need/potential for future research & 25 \itemline
\category{3}{Text}
& $< 6$ pages & 5 \itemline
% & Writing is original & 25 \itemline
& Properly cited sources & 15 \itemline
& Correct grammar, spelling, \newline and punctuation & 10 \itemline
\hline
\end{tabular}

Suggestions for improvement: 


\end{document}

